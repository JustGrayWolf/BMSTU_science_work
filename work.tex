\documentclass[12pt,a4paper]{report}
\usepackage[utf8]{inputenc}
\usepackage[english,russian]{babel}
\usepackage{indentfirst}
\usepackage{pdfpages}
\usepackage{titlesec}
\usepackage{listings}
\usepackage{amsmath}

% Вставка картинки
\usepackage{graphicx}
\graphicspath{{schemes/}}
\DeclareGraphicsExtensions{.pdf,.png,.jpg}

\usepackage[14pt]{extsizes}

\newcommand{\hsp}{\hspace{20pt}}
\titleformat{\chapter}[hang]{\large\bfseries}{\thechapter{. }}{0pt}{\large\bfseries}
\titlelabel{hlabel-formati}
\titlespacing{\chapter}{42pt}{-20pt}{12pt}
\titleformat{\section}[hang]{\large\bfseries}{\thesection{. }}{0pt}{\large\bfseries}
\titlespacing{\section}{42pt}{12pt}{5pt plus 5pt}

% Отступ абзаца
\usepackage{indentfirst}
\setlength{\parindent}{1.5cm}

% Межстрочный интервал
\usepackage{setspace}
\onehalfspacing % интервал 1.5

\usepackage[left=3cm, right=1cm, top=2cm, bottom=2cm]{geometry}

\begin{document}
% Титульник
%\includepdf[pages=1]{titul.pdf}
% Оглавление
\tableofcontents

\newpage
\chapter*{Введение}
\addcontentsline{toc}{chapter}{Введение}
Практически любой вид человеческой деятельности связан с ситуациями, когда име-
ется несколько возможностей и человек волен из этих возможностей выбрать любую, наибо-
лее подходящую ему.
Задачи наилучшего выбора изучает теория принятия решений. С ее помощью можно
научиться осуществлять выбор более обоснованно, эффективно используя имеющуюся в на-
личии информацию о предпочтениях. Эта теория помогает избежать принятия заведомо не-
годных решений и учесть возможные отрицательные последствия непродуманного выбора.

Наиболее обширной задачей многокритериального выбора является задача векторной оптимизации. Применение решений данной задачи может использоватся для экономических, проектных и даже научных задач, в которых нужно достич оптимального соотношения параметров.



Можно выделить следующие задачи научной работы:
\begin{itemize}
    \item Сформулировать задачу векторной оптимизации;
    \item изучить методы решения задачи;
    \item выбрать один из методов решения и исследовать его;
    \item составить заключение по проделанной работе.
\end{itemize}

\newpage
\chapter{Аналитическая часть}
В данном разделе будут сформулированы задачи многокритериального выбора и векторной оптимизации.
\section{Задача многокритериального выбора}
\section{Множество Парето}
\section{Задача векторной оптимизации}
\section{Способы решения задачи векторной оптимизации}
\renewcommand\bibname{Список литературы}
\addcontentsline{toc}{chapter}{Список литературы}
\makeatletter % список литературы
\def\@biblabel#1{#1. }
\makeatother
\begin{thebibliography}{2}
	\bibitem{} В. Д. Ногин ПРИНЯТИЕ РЕШЕНИЙ В МНОГОКРИТЕРИАЛЬНОЙ СРЕДЕ: Наука. Гл. ред. физ.-мат. лит. 2002.
    \bibitem{} А.Г. Коротченко Е.А. Кумагина В.М. Сморякова ВВЕДЕНИЕ В МНОГОКРИТЕРИАЛЬНУЮ ОПТИМИЗАЦИЮ: Гл. ред. Нижегородского государственного университета
им. Н.И. Лобачевского 2017
	\bibitem{} Подиновский В.В. Ногин В.Д. Парето-оптимальные решения многокритериальных задач. М.: Наука. Гл. ред. физ.-мат. лит. 1982.
	\bibitem{} Задачи векторной оптимизации [Электронный ресурс]. Режим доступа: https://poisk-ru.ru/s39497t19.html (дата обращения 10.12.2021)
	\bibitem{} Лекция. Векторная оптимизация. [Электронный ресурс]. Режим доступа: https://davaiknam.ru/text/lekciya-vektornaya-optimizaciya (дата обращения 12.12.2021)
\end{thebibliography}

\end{document}